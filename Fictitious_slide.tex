\documentclass[dvipdfmx,fleqn]{beamer}
%\documentclass[dvipdfmx,fleqn,handout]{beamer}
\usepackage{amsmath,amssymb,amsthm}

\mode<presentation>
{
  \usetheme{default}
}

\title{\Large Fictitious play}
\author{\large Yusuke Murakami}
\date{\small 28 Jun. 2014}

\usefonttheme{professionalfonts}

\setbeamercovered{transparent=20}

\setbeamertemplate{navigation symbols}{} 
\setbeamertemplate{footline}[frame number] 



\begin{document}

\sffamily
\gtfamily


\begin{frame}
  \titlepage
  \thispagestyle{empty}
\end{frame}

\setcounter{framenumber}{0}

\begin{frame}
\frametitle{Fictitous Playとは}
\begin{itemize}\setlength{\parskip}{0.5em}
\item
Ficititious playの前提

	\begin{itemize}\setlength{\parskip}{0.5em}
 	\item
	プレーヤーは$2$人.
	
	それぞれのプレーヤーをプレーヤー$0$、プレーヤー$1$とする.
	
	\item
	各プレーヤーには行動が$m$種類ある.
	
	(プレーヤーごとに行動の種類数に差があっても良い).
	
	それぞれの行動を行動$0$、行動$1$、$\cdots$、行動$m-1$とする.
	
	\item
	各プレーヤーにとって、その行動をとった時の利得が予め定まっている.
	それは利得行列の形で表される.
	
	\item
	期間は$0$期から$T$期までの繰り返しゲーム.
	
	\item
	各プレーヤーは、後に説明する「信念」を元に、
	その期に利得が最大になるような行動を選択する.
	利得が最大になる行動が複数ある場合はその中から
	等確率でランダムに選択する.
	
	\item
	将来利得を見据えた行動は取らない.
	
	\end{itemize}

\end{itemize}
\end{frame}


\begin{frame}
\frametitle{Fictitous Playとは}
\begin{itemize}\setlength{\parskip}{0.5em}

\item
記号

	\begin{itemize}\setlength{\parskip}{0.5em}
	\item $t$期にプレイヤー$i$が選択した行動は$a_i(t)$で表す.
	
	\item $t$期にプレイヤー$i$が持っている「信念」は$x_i(t)$で表す.
	
	\end{itemize}

\item
信念

	\begin{itemize}\setlength{\parskip}{0.5em}
	\item
	$t$期のプレーヤー$0$が持つ信念は、それまでの期間に相手
	(プレーヤー$1$)が取った行動の「平均」で表される.すなわち、
	\begin{equation}
	x_0(t) = \frac{a_1(0)+\cdots+a_1(t-1)}{t} \label{eq0}
	\end{equation}
	となる.
	
	\item
	プレーヤー$1$も同様に定式化する.
	
	\item
	初期の行動時には信念がないので、$x_0(0)$および$x_1(0)$はランダムに行動する.
	\end{itemize}

\end{itemize}
\end{frame}

\begin{frame}
\frametitle{今回の設定と漸化式}
\begin{itemize}\setlength{\parskip}{0.5em}

\item
今回はコードの都合で、いずれの行動をとっても利得が同じ場合は、
ランダムではなく若い数の行動をとることにした.

\item
また、今回はシミュレーションを面白くするために、信念を以下のようにした.
\begin{eqnarray} 
x_0(t) &=& \frac{x_0(0)+a_1(0)+\cdots+a_1(t-1)}{t+1} \label{eq1-0} \\
x_1(t) &=& \frac{x_1(0)+a_0(0)+\cdots+a_0(t-1)}{t+1} \label{eq1-1}
\end{eqnarray}

\item
\eqref{eq1-0}と\eqref{eq1-1}から、以下のような連立1階漸化式を考えることができる.
\begin{eqnarray} 
x_0(t+1) &=& x_0(t)+\frac{1}{t+2}(a_1(t)-x_0(t)) \label{eq2-0}\\
x_1(t+1) &=& x_1(t)+\frac{1}{t+2}(a_0(t)-x_1(t)) \label{eq2-1}
\end{eqnarray}


\end{itemize}
\end{frame}



\begin{frame}[containsverbatim]% verbatim 環境を使えるように
\frametitle{コードの説明}
\begin{itemize}\setlength{\parskip}{0.5em}

\item
初期読み込みと初期変数の設定

\begin{verbatim}
import matplotlib.pyplot as plt
import numpy as np
from random import uniform
fig, ax = plt.subplots()

T = 1000
s = ((1,-1),(-1,1),
     (-1,1),(1,-1))
\end{verbatim}

\verb|T|が期間、\verb|s|が利得行列(この場合はMatching Penniesゲーム).

\end{itemize}
\end{frame}

\begin{frame}[containsverbatim]% verbatim 環境を使えるように
\frametitle{コードの説明}
\begin{itemize}\setlength{\parskip}{0.5em}

\item
諸関数の設定

\begin{verbatim}
def ss(a, s):
    return ((s[0][a], s[1][a]),
            (s[2][a], s[3][a]))

def res(x, s):
    vec = (1-x, x)
    a = np.dot(s, vec) 
    return a.argmax()

\end{verbatim}

\verb|ss|はあるプレイヤーの利得行列を取り出す関数、
\verb|res|は最適反応を返す関数.
\verb|argmax()|を使っているので利得が同じ場合は若い数の行動を選ぶ.

\end{itemize}
\end{frame}


\begin{frame}[containsverbatim]% verbatim 環境を使えるように
\frametitle{コードの説明}
\begin{itemize}\setlength{\parskip}{0.5em}

\item
初期値の設定と漸化式の計算

\begin{verbatim}
x0 = [uniform(0, 1)]
x1 = [uniform(0, 1)]

s0 = np.array(ss(0, s))
s1 = np.array(ss(1, s))

for i in range(T-1):
    x0.append(x0[i]+(res(x1[i], s1)-x0[i])/(i+2))
    x1.append(x1[i]+(res(x0[i], s0)-x1[i])/(i+2))

\end{verbatim}

\verb|x0|と\verb|x1|に値を加えていく.
\verb|s0|と\verb|s1|はそれぞれの利得を関数\verb|s|で取り出している.

\end{itemize}
\end{frame}

\begin{frame}[containsverbatim]% verbatim 環境を使えるように
\frametitle{コードの説明}
\begin{itemize}\setlength{\parskip}{0.5em}

\item
グラフ化して保存、表示

\begin{verbatim}
ax.plot(x0, 'r-')
ax.plot(x1, 'b-')
plt.savefig('matchingpennies_plot.pdf', 
   bbox_inches='tight', pad_inches=0)
plt.show()
\end{verbatim}

グラフを書いて保存し、最後に表示する.

\end{itemize}
\end{frame}




\begin{frame}
\frametitle{図}
\begin{figure}
 \centering
 \includegraphics[scale=0.5]{matchingpennies_plot.pdf}
 \caption{matching penniesの場合の信念の推移}
 \label{fig:matchingpennies_plot}
\end{figure}
\end{frame}



\begin{frame}
\frametitle{matching penniesの考察}
\begin{itemize}\setlength{\parskip}{0.5em}
\item
最初は大きく信念が揺らぐが、
時間を経るにつれて徐々に収束して行く様子が分かる.

\item
相手の行動を追いかけるような挙動を見せている.

\item
収束先はどちらも$0.5$のように見える
\footnote{より詳しく見るには、このゲームを何度も繰り返し、
$T=1000$での頻度をヒストグラム化すると良い}.
これは行動$0$と行動$1$の混合戦略となり、Nash均衡に一致する.

\end{itemize}
\end{frame}

\begin{frame}
\frametitle{今後検討したいこと}
\begin{itemize}\setlength{\parskip}{0.5em}
\item
$T \to \infty$で行き着く先は必ず混合戦略も含めたNash均衡の1つになるのか?

	\begin{itemize}\setlength{\parskip}{0.5em}
	\item
	そもそも行動列は必ず収束するのか?
	
	\item
	\eqref{eq0}式と\eqref{eq1-0}式の間に差はないか?
	
	\end{itemize}

\item
$T \to \infty$で行き着く先が複数あるとき、その頻度に差はあるのか?

\item
頻度に差があったとすると、それにはどんな意味があるのか?

\item
プレーヤーを増やすとどうなるか?

\item
利得行列を確率変数にするとどうなるか?

\item
利得を考えるときに「信念」だけでなく「予想される将来利得(の割り引いたもの)」も
含めるとどうなるか?

\end{itemize}
\end{frame}




\end{document}